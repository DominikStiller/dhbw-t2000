\documentclass{scrreprt}


%----------------------------------------------------------------------------
% Package Includes
%----------------------------------------------------------------------------
\usepackage[USenglish]{babel}
\usepackage[T1]{fontenc}
\usepackage[utf8]{inputenc}
\usepackage{lmodern}

\usepackage{tocloft}
\usepackage{chngcntr}
\usepackage{amsmath}
\usepackage{mathtools}
\usepackage{interval}
\usepackage{graphicx}
\usepackage[hidelinks]{hyperref}
\usepackage{geometry}
\usepackage{textcomp}
\usepackage{siunitx}
\usepackage{setspace}
\usepackage{array}
\usepackage[outputdir=build]{minted}
\usepackage{tabularx}
\usepackage{makecell}
\usepackage{csquotes}
\usepackage[toc,nonumberlist,acronym,nopostdot]{glossaries}
\usepackage[
	backend=bibtex,
	bibwarn=true,
	bibencoding=utf8,
	sortlocale=en_US,
	urldate=long,
	style=ieee
]{biblatex}
\usepackage[activate={true,nocompatibility},final,tracking=true,kerning=true,spacing=true,factor=1100,stretch=10,shrink=10]{microtype}
%\usepackage{showframe}

%----------------------------------------------------------------------------
% Package Config
%----------------------------------------------------------------------------
% KOMA-Script
\KOMAoptions{
	pagesize=pdftex,
	twoside,		% Einseitiger Druck.
	parskip=half,		% Halbe Zeile Abstand zwischen Absätzen.
	headheight = 12pt,	% Höhe der Kopfzeile
	headsepline,		% Linie nach Kopfzeile.
	footsepline,		% Linie vor Fusszeile.
	footheight = 16pt,	% Höhe der Fusszeile
	abstract=true,		% Abstract Überschriften
	DIV=calc,			% Satzspiegel berechnen
	BCOR=20mm,			% Bindekorrektur links: 8mm
	headinclude=false,	% Kopfzeile nicht in den Satzspiegel einbeziehen
	footinclude=false,	% Fußzeile nicht in den Satzspiegel einbeziehen
	listof=totoc,		% Abbildungs-/ Tabellenverzeichnis im Inhaltsverzeichnis darstellen
	toc=bibliography	% Literaturverzeichnis im Inhaltsverzeichnis darstellen
}
\clubpenalty = 10000 % schließt Seitenumbruch nach der ersten Zeile eines neuen Absatzes aus
\widowpenalty = 10000 % schließt die letzte Zeile eines Absatzes steht auf einer neuen Seite aus
\displaywidowpenalty=10000

% graphicx
\graphicspath{ {./images/} }

% biblatex
\addbibresource{./misc/references.bib}

% Microtype
% http://www.khirevich.com/latex/micrsotype/
\SetProtrusion{encoding={*},family={bch},series={*},size={6,7}}
			     {1={ ,750},2={ ,500},3={ ,500},4={ ,500},5={ ,500},
	             6={ ,500},7={ ,600},8={ ,500},9={ ,500},0={ ,500}}
\SetExtraKerning[unit=space]
   				{encoding={*}, family={bch}, series={*}, size={footnotesize,small,normalsize}}
				{\textendash={400,400}, % en-dash, add more space around it
	               "28={ ,150}, % left bracket, add space from right
	               "29={150, }, % right bracket, add space from left
					\textquotedblleft={ ,150}, % left quotation mark, space from right
					\textquotedblright={150, }} % right quotation mark, space from left
\SetExtraKerning[unit=space]
				{encoding={*}, family={qhv}, series={b}, size={large,Large}}
				{1={-200,-200}, \textendash={400,400}}
\SetTracking{encoding={*}, shape=sc}{40}
\microtypecontext{spacing=nonfrench}

% glossary
\renewcommand*{\glsgroupskip}{}

% chngcntr
\counterwithout{figure}{chapter}
\counterwithout{table}{chapter}
\counterwithout{equation}{chapter}

% minted
\usemintedstyle{friendly}
\newminted{python}{
	mathescape,
	linenos,
	numbersep=5pt,
	frame=lines,
	framesep=2mm
}
\newmintinline{python}{}

% amsmath
\DeclareMathOperator*{\argmax}{arg\,max}
% https://tex.stackexchange.com/a/43009
\DeclarePairedDelimiter\abs{\lvert}{\rvert}
\makeatletter
\let\oldabs\abs
\def\abs{\@ifstar{\oldabs}{\oldabs*}}
\makeatother


%----------------------------------------------------------------------------
% Styling
%----------------------------------------------------------------------------

% Page
\geometry{margin=2.5cm, foot=1cm}

% Font
\KOMAoptions{fontsize=12pt}
\DeclareMathSizes{12pt}{14pt}{10pt}{10pt}
\setkomafont{title}{\Huge\textbf}
\onehalfspacing

% Table of Contents
\setcounter{tocdepth}{1}


%----------------------------------------------------------------------------
% Custom Commands
%----------------------------------------------------------------------------

\newcommand*{\vcenteredhbox}[1]{
	\begingroup
	\setbox0=\hbox{#1}\parbox{\wd0}{\box0}
	\endgroup
}

\newcommand{\multilinecell}[2][c]{%
	\begin{tabular}
		[#1]{@{}l@{}}#2
	\end{tabular}
}
